%
% Assignment 4a for CS3530 Computational Theory:
% Context-free Grammars and Pushdown Automata
% Fall 2017
%
% Problems taken from Sipser
%

\documentclass{article}

\usepackage[margin=1in]{geometry}
\usepackage{amsfonts}
\usepackage{amsmath}
\usepackage[english]{babel}
\usepackage[utf8]{inputenc}
\usepackage{ae,aecompl}
\usepackage{emp,ifpdf}
\usepackage{graphicx}
\usepackage{enumerate}

\ifpdf\DeclareGraphicsRule{*}{mps}{*}{}\fi

\empaddtoTeX{\usepackage{amsmath}}
\empprelude{input boxes; input theory}

% skip for paragraphs, don't indent
\parskip 6pt plus 1pt
\parindent=0pt
\raggedbottom

% a list environment with no bullets or numbers
\newenvironment{indentlist}{\begin{list}{}{\addtolength{\itemsep}{0.5\baselineskip}}}{\end{list}}

\begin{document}
\begin{empfile}

\begin{center}
\textbf{\Large CS 3530: Assignment 4a} \\[2mm]
Fall 2017
\end{center}

\raggedright

\section*{Exercise 2.5b (10 points)}

\subsection*{Problem}

Give informal descriptions and state diagrams of pushdown automata
for the languages in Exercise~2.4.

\begin{enumerate}
\item[\textbf{b.}] $\{w:w$ starts and ends with the same symbol$\}$
\end{enumerate}

\subsection*{Solution}

Context free grammar:\newline
$S$ $\rightarrow$ 0$A$0 | 1$A$1 | 0 | 1 \newline
$A$ $\rightarrow$ 0$A$| 1 $A$| $\varepsilon$\newline
Informal Description:\newline 
the grammar above gives the roles to $L$, where $S$ can be either 00, 11, 0 or 1 and $A$ can lead to 0 or 1 or $\varepsilon$.\newline
if the last symbol mathces the one on the stack then no input strings will be accepted. \newline
Let $q_0$ be the start state and $q_2$ is an accepted states, we can get the following pushdown automata.\newline
($q_0$,$\varepsilon$, $\varepsilon$) $\rightarrow$ ($q_0$,\$)\newline
Let B be equal to\newline 
($q_0$,1,$\varepsilon$)$\rightarrow$ ($q_2$,$\varepsilon$) \newline
($q_0$,0,$\varepsilon$)$\rightarrow$ ($q_2$,$\varepsilon$) \newline
($q_1$,0,$\varepsilon$)$\rightarrow$ ($q_1$,$\varepsilon$) \newline
($q_1$,1,$\varepsilon$)$\rightarrow$ ($q_1$,$\varepsilon$) \newline
Let A be equal to\newline
($q_0$,0,$\varepsilon$)$\rightarrow$ ($q_1$,0) \newline
($q_0$,1,$\varepsilon$)$\rightarrow$ ($q_1$,1) \newline
Let C be equal to\newline
($q_1$,0,0)$\rightarrow$ ($q_2$,$\varepsilon$) \newline
($q_2$,1,1)$\rightarrow$ ($q_2$,$\varepsilon$) \newline

($q_2$,$\varepsilon$,\$)$\rightarrow$ ($q_2$,$\varepsilon$) \newline

\begin{emp}(10,0)
  t := 2cm;
  node.q0("0"); q0.c = origin;
  node.q1("1"); q1.c = q0.c + (0,-t);
node.q2("2"); q2.c = q0.c + (0,-2t);
  makestart(q0);
  makefinal(q2);
  % draw the nodes
  drawboxed(q0,q1,q2);
  
  edge(q0,q1,1,btex A etex);
  edge(q1,q2,1,btex C etex);
  edge(q0,q2,45,btex B etex);
loop(q1,left, btex B etex);


\end{emp}



\section*{Exercise 2.6b (10 points)}

\subsection*{Problem}

Give context-free grammars generating the following languages.

\begin{enumerate}
\item[\textbf{b.}] The complement of the language $\{a^n b^n:n\geq 0\}$
\end{enumerate}

For all CFGs, describe the role that each rule performs as well as
giving the actual rule.

\subsection*{Solution}

Let $L_1$ be the compliment of the given language L.
and let $L_1$ = \{$a^n$ $b^m$ : $n$ $\neq$ $m$ \} $\cup$ \{($a$ $\cup$ $ b$)*$ba$($a$ $\cup$ $b$)*\}.\newline
we can start with the left half of the language to generate the fre grammar for it.\newline
$S_1$ $\rightarrow$ $a$ $S_1$ $b$ | $T$ | $U$ \newline
$T$ $\rightarrow$ $aT |a$\newline
$U$ $\rightarrow$ $Ub|a$ \newline
Let $S_2$ be the other half of the compliment language of $L$.\newline
$S_2$ $\rightarrow$ $RbaR$ \newline
$R$ $\rightarrow$ $RR$ | $a$ | $b$ | $\varepsilon$
\end{empfile}
\immediate\write18{mpost -tex=latex \jobname}
\end{document}
